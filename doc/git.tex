\chapter{Building CyDAQ from the SVN Repository}
\label{chap:svn}
This chapter describes in detail the steps required to obtain the \ADAQ
source code from its SVN repository home and then build the \ADAQ
executables from source.

\section{Prerequisites}
\label{sec:prereqs}
Before building the \ADAQ source code, a number of dependencies must be
installed on the user's local system in order to ensure a successful
build of the \ADAQ source code. While the specific details of building
each of these dependencies is outside the scope of this manual, a few
notes are provided below. It should be noted that all required
software dependencies are already installed on the Ionetix server,
greatly accelerating the process of building and the using the \ADAQ
software.

\subsection{\ROOT}
If the user will be building \ADAQ on his/her local Linux machine, \ROOT
may be downloaded from the following location:
\purl{http://root.cern.ch/drupal/content/downloading-root}. The most
recent version of \ROOT that has been successfully tested with the \ADAQ
source code is version 5.32.01. While \ROOT binaries are available for
direct installation, it is preferable to build \ROOT from source code
to ensure perfect configuration to the user's local environment. The
user is recommended to download the appropriate tarball and then
follow the directions for installing \ROOT from source that can be
found here:
\purl{http://root.cern.ch/drupal/content/installing-root-source}

If the user will be building \ADAQ on the Ionetix server, \ROOT version
5.32.01 has already been installed in
\texttt{/usr/local/root/root\_v5.32.01}. To use this version of \ROOT,
the user needs to add the following lines to his/her \texttt{.bashrc}
file:
\begin{lstlisting}
  export ROOTSYS=/usr/local/root/root_v5.32.01/root
  export PATH=$ROOTSYS/bin:$PATH
  export LD_LIBRARY_PATH=$ROOTSYS/lib:$LD_LIBRARY_PATH
\end{lstlisting}
These environmental variables ensure that the user has full access to
the \ROOT binaries and libraries required to build and run the \ADAQ
source code.


\subsection{\texttt{Boost}}
If the user will be building \ADAQ on his/her local Linux machine, the
\BOOST C++ libraries can either be installed via the operating systems
package manager (\texttt{apt-get} on Ubuntu or \texttt{yum} on
Fedora/RedHat) or downloaded from the \BOOST homepage at
\purl{http://www.boost.org/}. The vast majority of \BOOST
functionality is conventiently provided by header files and therefore
does not require building from source; such is the case for the parts
of \BOOST used by \ADAQ. The user can follow directions for installation
of the most recent \BOOST release (1.49.0) on Linux machines here:
\purl{http://www.boost.org/doc/libs/1_49_0/more/getting_started/unix-variants.html}

if the user will be building \ADAQ on the Ionetix server, \texttt{Boost}
is already installed at \texttt{/usr/include/boost}. No special action
is required of the user; \texttt{Boost} is ready to be used.


\section{Getting the \ADAQ source code}
\label{sec:sourcecode}
The \ADAQ source lives permanently in a Subversion code management
repository on the Ionetic Linux server \footnote{For the uninitiated,
  a good code management system is a highly scalable piece of software
  that allows developers to efficiently, seamlessly, and confidently
  work on a software project together with the full history of
  everyone's changes to the project (as well as all of the original
  files) fully documented and accessible at any time. They are
  incredibly useful for a developers or even just a single developer,
  if for nothing other than to document all changes and provide the
  ability to go back to any version of any file at any time.}. While
anyone with a user account on the Ionetix Linux server may check out a
working copy of the code, only users who are in the ``CyDAQ'' group
have permission to check in changes to the repository. To be added to
the CyDAQ developers group, please contact the author at
\texttt{hartwig@psfc.mit.edu}.

Checking out a copy of the source code depends on the location from
which you connecting to the SVN repository: a ``local'' connection
refers to a user who is logged into and working on the Ionetix server;
a ``remote'' connection refers to a user who is working on a remote
computer with SSH capababilites that is running Subversion.\\

\noindent
To checkout a copy of the \ADAQ source code \textit{locally}, the user
should open a terminal on the Ionetix server and type:
\begin{lstlisting}
  svn co file:///usr/local/svn_repos/__CYDAQ_REPO__ /path/to/checkout
\end{lstlisting}
where the user should replace \texttt{/path/to/checkout} with the
directory
he/she would like SVN to place the checked out \ADAQ source code.\\

\noindent
To checkout a copy of the \ADAQ source code \textit{remotely}, the user
should open a terminal on his/her local machine and type:
\begin{lstlisting}
  svn co svn+ssh://username@ipaddress/usr/local/svn_repos/__CYDAQ_REPO__ /path/to/checkout
\end{lstlisting}
where the user should replace \texttt{username} with his/her username
on the Ionetix server, \texttt{ipaddress} with the IP address of the
Ionetix server, and /path/to/checkout with the directory on the
user's local machine that he/she would like SVN to place the checked
out \ADAQ source code.

\section{Overview of \ADAQ directory structure}
\label{sec:directorystructure}
After checking out the \ADAQ source code from the SVN repository as
described in Section~\ref{sec:sourcecode}, navigate to the top-level
\ADAQ directory, which will be the path specified to SVN when the copy
of the source code was checked our of the repository
(/path/to/checkout in the directions above). This directory is known
as the top-level directory. The directory and file structure below
this directory is descibed below:

\begin{itemize}
  \item{\textbf{analysis/}: contains a ``template'' for offline data
    analysis of digitized waveforms produced with
    \texttt{CyDAQRootGUI}. Designed to show the user how to access
    stored waveforms and provide a basis for his/her own analysis
    code.}
  \item{\textbf{manual/}: contains all of the files (\LaTeX files,
    images files in the images/ subdirectory, and the GNUmakefile for
    creating the beautiful manual you are now reading.}
  \item{\textbf{source/}: contains the bulk of the \ADAQ source
    code. The \textbf{versions/} subdirectory (empty as present) will
    contain snapshots of the \ADAQ source. The snapshots will be
    versioned and used when acquiring mission critical data, such that
    the state of the \ADAQ source code can be linked to experimental
    data sessions. The \textbf{trunk/} subdirectory contains the
    \texttt{CyDAQRootGUI} source code in subdirectories and the GNU
    makefile:
    \begin{itemize}
      \item{\textbf{bin/}: contains the \texttt{CyDAQRootGUI} binary
        after building via the GNU makefile. Other useful binaries are
        also in this directory.}
      \item{\textbf{build/}: a dumping ground for transient files
        created during the build process. The files are source code
        object files and \ROOT dictionary files.}
      \item{\textbf{include/}: C++ header files for
        \texttt{CyDAQRootGUI}.}
      \item{\textbf{lib/}: CAEN libraries required to build and run
        \texttt{CyDAQRootGUI}. The purpose of including these
        libraries here (rather than relying on installed CAEN
        libraries in places like /usr/lib) is for portability and to
        maintain explicit control over the versioning of CAEN
        libraries. The CAEN libraries are evolving and only somewhat
        decently documented, so it is important to ensure the correct
        versions of the libraries are used. Subdirectories contain
        libraries and soft links for both 32- and 64-bit architectures
        in the \textbf{x86/} and \textbf{x86\_64/} subdirectories.}
      \item{\textbf{src/}: C++ source code for \texttt{CyDAQRootGUI}.}
    \end{itemize}
  }
  \item{\textbf{utilities/}: contains some useful Python scripts.}
\end{itemize}

With the \ADAQ source code checked out and a basic understanding of the
source code structure, it's time to build the \texttt{CyDAQRootGUI}
binary.

    
\section{Configuring your environment}
\label{sec:config}
The \ADAQ source requires several environmental variables to be defined
in order to build and run correctly.
\begin{itemize}
\item{\textbf{CYDAQHOME}: Used by the \ADAQ GNU makefiles to determine
    the full paths to a number of source files and libraries. The
    value should be set as the full path to the top level of the \ADAQ
    source code.}
\item{\textbf{HOSTTYPE}: Used by the \ADAQ GNU makefiles to build the
    \ADAQ software tools for the correct architecture. The value should
    be set to ``x86'' for 32-bit architecture and ``x86\_64'' for
    64-bit architectures.}
\item{\textbf{PATH}: The Linux environmental variable that contains
    directories that are searched to determine the location of
    binaries. The bin directory for CyDAQRootGUI should be added to
    the \texttt{PATH} variable.}
\item{\textbf{LD\_LIBRARY\_PATH}: The Linux environmental variables that
    contains directories that are searched for dynamically linked
    libraries. The \ADAQ source code contains most of the required
    libraries for both 32- and 64-bit architectures. The lib directory
    for CyDAQRootGUI should be added to the \texttt{LD\_LIBRARY\_PATH}
    variable.}
\end{itemize}

The easiest way to set these four variables is to add the required
lines to the user's \texttt{.bashrc} file (which exists in his/her
home directory). For example, consider that user \texttt{zaphod} has
checked out a copy of the \ADAQ source code to the directory
\texttt{/home/zaphod/CyDAQ\_CheckOut} of the Ionetix server, which runs
a 64-bit operating system. He would need to add the following lines to
the file \texttt{/home/zaphod/.bashrc}:
\begin{lstlisting}
  export CYDAQHOME = /home/zaphod/CyDAQ_CheckOut
  export PATH = $CYDAQHOME/source/trunk/bin:$PATH
  export HOSTTYPE = x86_64
  export LD_LIBRARY_PATH = $CYDAQHOME/source/trunk/lib/$HOSTTYPE:$LD_LIBRARY_PATH
  $
\end{lstlisting}
\noindent
After these lines are added to the \texttt{.bashrc} file, the user
will either need to source the file in the currently open terminal or
simply close the terminal window and reopen a new one for the
environmental variables to take effect.

\section{Building the \texttt{CyDAQRootGUI} binary}
After ensuring that the prequisites described in
Section~\ref{sec:prereqs} have been successfully installed and
that the user's environment is configured as described in
Section~\ref{sec:config}, the user is ready to build the
\texttt{CyDAQRootGUI} binary.
\begin{itemize}
\item{Navigate to the \texttt{CyDAQRootGUI} directory:
    \begin{lstlisting}
      cd CYDAQHOME/source/trunk
    \end{lstlisting}
  }
\item{Build the binary from the source code using the provided GNU
    makefile:
    \begin{lstlisting}
      make
    \end{lstlisting}
    The GNU makefill will show the compilation steps and update the
    user on the progress of the build.
  }
\item{If successful, a new executable binary named
    \texttt{CyDAQRootGUI} should exist in the
    \texttt{\$CYDAQHOME/source/trunk/bin} directory
  }
\end{itemize}
That's all it takes! The user is now ready to begin running the
\texttt{CyDAQRootGUI} and should proceed to Chapter~\ref{chap:run}.
